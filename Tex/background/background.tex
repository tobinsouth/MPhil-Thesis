%!TEX root = ../thesis.tex
\chapter{Background \label{ch:background}}

\subsection{Introduction}

\subsubsection{Natural Language Processing} 

Natural language processing (NLP) is the area of study in which 'natural' human language is examined via machine. Natural language refers to either spoken or written language, designed to be understandable to a human listener or reader. This language is not explicitly designed to be machine understandable, and machine comprehension of this language is a challenging problem \todo{cite: the challenges of NLP}.

NLP is a broad term covering many models and techniques to computationally extracting meaningful information from text, ranging from the simple extraction of individual words, to the extraction of deeper semantic meaning. 

Early work in NLP focused around simple grammatical rules and small vocabularies, such as the work of Georgetown-IBM \todo{cite: John Hutchins. From first conception to first demonstration: the nascent years of machine translation, 1947–1954. a chronology. Machine Translation, 12(3):195–252, 1997.} to translate 60 sentences from Russian to English in 1954. With the rapid increase in computational power and digital text corpuses, modern NLP has focused or deeper challenges of extracting meaning from text with tools such as Word2Vec \todo{cite: word to vec} or deep learning methods such as Google's BERT \todo{cite: BERT}.

These methods face a daunting challenge, language is not only complex and often duplicitous, but contextual and ever-changing. \todo{end better}



\subsection{Entropy Basics}

