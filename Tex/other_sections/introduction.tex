%!TEX root = ../thesis.tex
\chapter{Introduction}\label{ch:introduction}

\section{Motivation}
% “A good newspaper, I suppose, is a nation talking to itself.”
% ― Arthur Miller

% \epigraph{\em The speed of communications is wondrous to behold. It is also true that speed can multiply the distribution of information that we know to be untrue.}{Edward R. Murrow, journalist (1908–1965)}


% News used to come too late; now it comes too early. -Marty Rubin

% Importance of media
Journalism has always been a cornerstone of democracy. It both shapes and is shaped by public sentiment, influencing the social sphere through its choices of content and its framing of issues. This `fourth estate' has turned the tides of history, with countless examples from the role of newspapers in distributing information during the American Revolution~\cite{tranFourthEstateFinal2016} to the exposé of the Panama Papers~\cite{odonovanValueOffshoreSecrets2019}. This influence is a perilous task; during the late 1990s the news-media played an important role in rapidly proliferating the misinformation linking the MMR vaccine to autism -- but also played a key role in exposing the fraudulent science afterwards~\cite{godleeWakefieldArticleLinking2011}. News-media can act as an powerful tool in boosting morale during wartime, and can equally serve as a machine of propaganda~\cite{jones2001censorship}. News-media must fight many challenges to maintain credibility, profitability and journalistic integrity. 


% The rise of internet media
The rise of the internet and online social media sites has increased some of these challenges. New mediums for online engagement have induced a rapid change in the ways citizens consume content. Roles the media has historically played have become decentralised, with social movements such as the Arab Spring and Occupy Wall Street arising in large part through collective action of citizens online~\cite{skinnerSocialMediaRevolution2011}. This decentralised collective system of news is a powerful connector, but also prone to increasing polarisation~\cite{barberaTweetingLeftRight2015}. %
% This is exemplified by the rise of algorithmically amplified misinformation~\tocite{The spread of true and false news online}. 
Despite these changes, traditional news-media has adapted to, and in many cases thrived in, this new environment. Many popular news-media organisations are primary funded through digital advertisements and digital-only subscriptions, including the New York Times where digital revenue overtook print for the first time in 2020~\cite{tracyNewYorkTimes2021}. Traditional news-media continues to produce a significant amount of content, which it typically attempts to disseminate over social media and through online platforms as well as offline.


% Techniques to analyse digital news
This huge volume of content, both traditionally produced and user generated, makes analysis difficult. Qualitative approaches can study impactful events and articles, but cannot encompass the deluge of content that played smaller roles in a story's development. Conversely, quantitative tools are often capable of spanning the totality of relevant content, but often focus on single aspects of a news story, such as counts of words or sets of words over time~\cite{michelQuantitative2011,patriciaChanging2013, pechenick2015characterizing}. These quantitative approaches are important lenses through which to view news, but need to be combined with context. Where context and quantitative analysis of news are combined, novel perspectives can be formed; exemplifying this is the usefulness of hyperlinks in revealing the cross-linguistic communication between bloggers during the Haitian earthquake in 2010~\cite{haleNetIncreaseCrossLingual2012}. 
% add another example


% More tools are needed 
In particular the study of \emph{information diffusion} has benefited from the internet and online social media, with studies into topics such as information propagation in the blogspace~\cite{gruhlInformationDiffusionBlogspace2004} and the spread of true and false news through social media~\cite{vosoughiSpreadTrueFalse2018}. The study of information diffusion has a longer history using other contexts such as  the spread of messages in organizations~\cite{wuInformationFlowSocial2004}, and the information flow within research and development laboratories~\cite{allenInformationFlowResearch1969}. These studies of diffusion are often accompanied by models of flow, usually in the form of epidemiological or statistical physics models~\cite{castellanoStatisticalPhysicsSocial2009}. These models and empirical studies share a common theme; they focus on the transmission and diffusion of singular ideas or packets of information. 


% an information theoric appraoch
The question we study here is of information flow more generally. Rather than constraining the analysis to individual stories or discrete packets of information, we use tools from information theory that capture the complexity of language in this flow. Similar approaches have used information-theoretic tools on the temporal data from social media~\cite{versteegInformationTransferSocial2012} and more recent work has used non-parametric entropy estimators to study information flow in a Twitter user's local social network~\cite{bagrowInformationFlowReveals2019}. These approaches have subsequently been extended into a model of information flow built from these simple estimators~\cite{bagrow_quoter_2018,pondComplexContagionFeatures2020}. 

This thesis significantly extends these approaches by speeding up and validating the convergence of these non-parametric entropy estimators on text, extending the estimators into several measures which better estimate and represent information flow, and applying these tools to the novel context of news-media data. These new additions to the literature contribute to the study of information flow in news, but also provide significant improvements to the toolkit of information flow analysis in any systems containing connected producers of textual information.


\subsection{Preliminary background}
% My greatest concern was what to call it. I thought of calling it 'information,' but the word was overly used, so I decided to call it 'uncertainty.' When I discussed it with John von Neumann, he had a better idea. Von Neumann told me, 'You should call it entropy, for two reasons. In the first place your uncertainty function has been used in statistical mechanics under that name, so it already has a name. In the second place, and more important, no one really knows what entropy really is, so in a debate you will always have the advantage.'
% Scientific American (1971), volume 225, page 180.
% Explaining why he named his uncertainty function "entropy".

The rise of modern news is intricately linked to the rapid changes in our tools of information technology. These technologies are useful not only in providing information to the public, but also in analysing it~\cite{shannon_mathematical_1948,cover_elements_2012}. In contrast to traditional studies of news, techniques built from fundamental engineering tools can provide an investigation into news independently of the broader qualitative context, using only the data.   

This notion of information is drawn from the field of information theory established in the first half of the 20th century~\cite{shannon_mathematical_1948,nyquist1924certain,hartley1928transmission}. Information in this context takes human interpretable content and quantifies it into useful computational objects. This approach allows for an analysis of these objects through measuring uncertainty, predictability and transmission. These quantities, although summary statistics of the full informational context, can be done at speeds orders of magnitude faster than traditional human comprehension, allowing for large quantities of information to be analysed quickly. 
% An example of these tools we will strongly reply on is non-parametric entropy rate estimation.

In this work, these large quantities of information will come in the form of human readable text. To convert human-style language into discrete quantified information we draw upon several techniques from natural language processing. These techniques allow us to both convert natural language text into sequences of consistent `tokens', but also to generate synthetic text that allows for model evaluation using simulations. 

Combining these two tools allows us to create networks -- mathematical constructions where objects called `nodes' are connected by directed relationships called `edges'~\cite{newman_networks_2018}. These constructed networks allow us to begin examining the system as a whole using analysis tools on networks. The tools will include a variety of ranking methods drawn from centrality measures, sports ranking, and graph-topological arrangement.


\subsection{Outline of thesis}

% \epigraph{ \em ``The worst thing you can do is to completely solve a problem.''}{Dan Kleitman}

% The ultimate goal of mathematics is to eliminate any need for intelligent thought. — Alfred N. Whitehead

This thesis is separated into five main chapters. We begin \autoref{ch:background} by providing a background of the three fields mentioned above: information theory, networks, and natural language processing. We introduce several important concepts, such as the notion of entropy rate, cross entropy rate and maximal predictability.

In \autoref{ch:data} we introduce our data and describe how we collected and cleaned it. This data, representing a year's worth of tweets from 154 news-media organisations, is representative across the US political spectrum and contains a total of 2,977,980 tweets. We include an exploratory data analysis of this corpus.

In \autoref{ch:crossentropy} we develop a new modified cross entropy rate estimator by extending the work of Kontoyianni \emph{et al.}~\cite{kontoyiannis_nonparametric_1998} and Bagrow \emph{et al.}~\cite{bagrowInformationFlowReveals2019}. The convergence of this estimator is studied using both simulated and real text to confirm it as a useful tool for finding time-based entropy relationships between text sequences. This estimator is optimized and released in an open source Python package listed in \autoref{app:code}.

In \autoref{ch:quotermodel} we evaluate the use of the cross entropy rate estimator as a measure of information flow and discover it is incomplete. We hence extend the estimator into a collection of possible flow measures that combine both directions of cross entropy with self-entropy rates and local network information to formulate a set of measures to estimate information flow. These measures are evaluated using simulated models of networked information flow which encapsulate the complexity of natural language combined with chains and cycles of artificial quoting. Measures that normalised differences in cross entropy flow by target information content prove most effective, with a single measure based on local neighbour information consistently outperforming all other measures.

Finally, the highest performing measure is then applied the news data previously collected and cleaned to construct a network of information flows in news. This network is analysed in \autoref{ch:ranking} to examine the feasibility of a definitive ranking. Several ranking methods are used including network centralities, a sports ranking and a graph topological approach. The network of flows is shown to have centralities and ranks that are sensitive to small changes in the network, and the diverse rankings are highly discordant. This diversity of results highlights key difficulties of simplifying such a network. 

We conclude with a discussion and outlook for further work in \autoref{ch:conclusion}.


